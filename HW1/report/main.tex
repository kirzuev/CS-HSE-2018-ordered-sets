\documentclass{article}
\usepackage[utf8x]{inputenc}
\usepackage[russianb]{babel}
\usepackage[usenames]{color}
\usepackage{vmargin}
\usepackage{ amssymb }
\usepackage{listings}
\usepackage[linesnumbered,boxed]{algorithm2e}
\usepackage{graphicx}
\usepackage{amsthm}
\usepackage{setspace}
\usepackage{indentfirst}
\usepackage[noend]{algorithmic}
\onehalfspacing
\setpapersize{A4}
\setmarginsrb{3cm}{2cm}{3cm}{2cm}{0pt}{0mm}{0pt}{13mm}
\sloppy
\renewcommand\contentsname{Оглавление}
\renewcommand{\labelenumi}{\theenumi)}

\begin{document}

\null\hfill\begin{tabular}[t]{l@{}}
	\textit{Kirill Zuev}
\end{tabular}

\begin{center}
	\textbf{Home assignment № 1}
\end{center}

\textbf{Task 1.}

$U$ is a finite set, $f: U \to U$ and $f$ is a surjective function. It means that
$$\forall y \in U~\exists x \in U: f(x) = y$$

Let $f$ is \textbf{not} an injective function. Then $\exists x_1 \neq x_2: f(x_1) = f(x_2)$. And also $\exists y'~\nexists x: f(x) = y'$. But $U$ is finite set and $\forall y \in U~\exists x \in U: f(x) = y$. Therefore,
$$\forall x_1, x_2: f(x_1) = f(x_2) \Rightarrow x_1 = x_2 $$
It means that $f$ is also an injective function.\\

\textbf{Task 2.}

a) Asymmetric and transitive: 4231\\
Computate this number by a Haskell program \textit{TransitiveAsymmetricRelations.hs}.\\

b) Antisymmetric and antireflexive: $3^{10}$\\
Consider the relation matrix $P$. Diagonal elements $p_{ii}$ $(i = 1 \dots 5)$ must be always 0. For non-diagonal elements:
$$\forall i, j = 1 \dots 5,~i \neq j: (p_{ij} = 1 \land p_{ji} = 0) \lor (p_{ij} = 1 \land p_{ji} = 0) \lor (p_{ij} = 0 \land p_{ji} = 0) $$
There are 10 pairs of different set elements and all of them have 3 of 4 variants of values. Therefore, we have a result: $3^{10}$.\\

\textbf{Task 3.}

In Task 2 I wrote a Haskell program (\textit{TransitiveAsymmetricRelations.hs}) including a function \textit{isTransitive}. This function get a relation matrix $P$ ($n \times n$), square it ($O(n^3)$) and check that result of multiplication is including on original relation matrix $P^2 \subseteq P$ ($O(n^2)$). As result we have a complexity: $O(n^3 + n^2) = O(n^3)$.\\

\textbf{Task 4.}

Another my Haskell program (\textit{TopologicalSort.hs}) include topological sort algorithm with two types of input data: list of edges (a) and adjacency matrix (b).\\

a) List of edges complexity: $O(n+m)$

b) Adjacency matrix complexity: $O(n^2)$

where $n$ --- number of vertices, $m$ --- number of edges.\\

\textbf{Task 5.}

$R: (x_1, y_1) R (x_2, y_2) \leftrightarrow x_1 \leq x_2, y_1 \leq y_2$.\\

$R$ is reflexive.\\
$\forall (x,y) \in Z^2: x \leq x,~y \leq y$, because $\leq$ is reflexive $\Rightarrow (x,y) R (x,y)$.\\

$R$ is transitive.\\
$\forall (x_1,y_1), (x_2,y_2), (x_3,y_3): (x_1,y_1) R (x_2,y_2), (x_2,y_2) R (x_3,y_3) \Rightarrow x_1 \leq x_2,~y_1 \leq y_2,~x_2 \leq x_3$,\ $y_2 \leq y_3$. $\leq$ is transitive $\Rightarrow x_1 \leq x_3,~y_1 \leq y_3 \Rightarrow (x_1,y_1) R (x_3,y_3)$.\\

$R$ is antisymmetric.\\
$\forall (x_1,y_1), (x_2,y_2): (x_1,y_1) R (x_2,y_2), (x_2,y_2) R (x_1,y_1) \Rightarrow x_1 \leq x_2,~y_1 \leq y_2,~x_2 \leq x_1,~y_2 \leq y_1$. $\leq$~is~antisymmetric $\Rightarrow x_1 = x_2,~y_1 = y_2 \Rightarrow (x_1,y_1) = (x_2,y_2)$.\\

It means that $R$ is a partial order.\\

1) $A_1 = {(x,y)~|~x \leq 3,~y \leq 4}$

$\nexists min,~max = (3,4)$.\\

2) $A_2 = {(x,y)~|~x^2 + y^2 \leq 4}$

$min = (-2,-2),~max = (2,2)$.

 \end{document}